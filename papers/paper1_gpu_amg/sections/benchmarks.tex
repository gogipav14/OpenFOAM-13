\section{Benchmark Cases}
\label{sec:benchmarks}

All benchmarks use OpenFOAM~13 with the OGL module compiled against
Ginkgo~v1.11.0 (CUDA backend, SM~12.0). The hardware platform is an NVIDIA
RTX~5060 GPU (\SI{8}{\giga\byte} GDDR7 VRAM) paired with an Intel desktop CPU.
Docker containers ensure reproducible builds from source.

\subsection{Case~A: 2D lid-driven cavity (scaling study)}
\label{sec:cavity}

The primary scaling vehicle is a 2D lid-driven cavity with $k$--$\varepsilon$
RANS turbulence, solved with PISO (3~pressure correctors) for 20~timesteps.
Meshes range from $200 \times 200$ (\num{40000}~cells) to $2236 \times 2236$
(\num{5e6}~cells) across eight resolutions.

Four solver configurations are compared:
\begin{enumerate}
    \item \textbf{CPU baseline}: GAMG + Gauss--Seidel smoother for pressure;
        smoothSolver + symmetric Gauss--Seidel for momentum.
    \item \textbf{GPU-p}: OGL PCG + AMG for pressure; CPU solvers for
        momentum.
    \item \textbf{GPU-all (BJ)}: OGL PCG + AMG for pressure; OGL BiCGStab
        + Block-Jacobi for momentum.
    \item \textbf{GPU-all (ILU)}: OGL PCG + AMG for pressure; OGL BiCGStab
        + ILU-ISAI for momentum.
\end{enumerate}

This case features a uniform structured mesh and constant transport properties,
representing a best-case scenario for GPU AMG: the sparsity pattern is regular,
coefficients are well-conditioned, and the pressure Laplacian is nearly constant
across corrector steps.

\subsection{Case~B: 3D stirred-tank bioreactor (Sartorius 50L)}
\label{sec:sartorius}

To evaluate industrial relevance, we benchmark a 3D stirred-tank bioreactor
(Sartorius 50L geometry) with a Rushton turbine impeller modeled via the
Multiple Reference Frame (MRF) approach. The vessel is
$\SI{0.41}{\metre} \times \SI{0.41}{\metre} \times \SI{0.465}{\metre}$ and
meshed with \texttt{blockMesh} at six resolution levels from
\num{73600} to \num{5.1e6}~cells (Table~\ref{tab:sartorius-meshes}).

\begin{table}[htbp]
\centering
\caption{Sartorius 3D mesh levels.}
\label{tab:sartorius-meshes}
\begin{tabular}{@{}lrrr@{}}
\toprule
Label & $n_x \times n_y \times n_z$ & Cells & Note \\
\midrule
73k   & $40 \times 40 \times 46$    & \num{73600}   & Base \\
200k  & $60 \times 60 \times 56$    & \num{201600}  & \\
500k  & $80 \times 80 \times 80$    & \num{512000}  & \\
1M    & $100 \times 100 \times 100$ & \num{1000000} & \\
2M    & $126 \times 126 \times 126$ & \num{2000376} & \\
5M    & $172 \times 172 \times 172$ & \num{5088448} & \\
\bottomrule
\end{tabular}
\end{table}

The solver is \texttt{incompressibleFluid} with $k$--$\varepsilon$ RANS
turbulence, PIMPLE with 2~outer correctors and 2~inner pressure correctors
(4~pressure solves per timestep), run for 10~timesteps ($\Delta t =
\SI{0.005}{\second}$, 40~total pressure solves). Three solver configurations
are compared:
\begin{enumerate}
    \item \textbf{CPU GAMG}: GAMG + Gauss--Seidel for pressure; smoothSolver +
        symmetric Gauss--Seidel for momentum.
    \item \textbf{GPU AMG + BJ}: OGL PCG + AMG for pressure; OGL BiCGStab +
        Block-Jacobi for momentum.
    \item \textbf{GPU AMG + ILU}: OGL PCG + AMG for pressure; OGL BiCGStab +
        ILU-ISAI for momentum.
\end{enumerate}

This case presents several challenges absent from Case~A:
\begin{itemize}
    \item \textbf{Variable coefficients}: The MRF source terms and turbulent
        viscosity create spatially varying pressure Laplacian coefficients that
        change between PIMPLE outer correctors.
    \item \textbf{3D unstructured zones}: The \texttt{topoSet}-defined rotating
        zone creates an effectively unstructured connectivity pattern.
    \item \textbf{Mixed boundary conditions}: Walls, baffles, and the MRF
        interface produce a heterogeneous matrix structure.
\end{itemize}
These factors stress-test the AMG hierarchy quality and the cost--benefit
balance of hierarchy caching.
