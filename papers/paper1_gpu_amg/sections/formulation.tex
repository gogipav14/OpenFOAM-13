\section{Mathematical Formulation}
\label{sec:formulation}

\subsection{Pressure Poisson system}
\label{sec:pressure}

The pressure equation arising from the PISO/PIMPLE algorithm yields a
symmetric positive-definite linear system $Ap = b$, where $A$ is the
Laplacian operator discretized on the finite-volume mesh. We solve this
system with preconditioned conjugate gradients (PCG) using algebraic
multigrid (AMG) as the preconditioner.

OpenFOAM stores the matrix in Lower-Diagonal-Upper (LDU) face-based format.
For GPU execution, we convert to Compressed Sparse Row (CSR) format. The
sparsity structure is static on fixed meshes, so we cache the CSR column
indices and row pointers, updating only the numerical values each solve
(\texttt{invalidateValues()} refresh).

\subsubsection{AMG preconditioning}

We use Ginkgo's Parallel Graph Matching (PGM) algebraic multigrid with
Jacobi smoothing. The PGM coarsening algorithm constructs the multigrid
hierarchy through parallel graph matching of the adjacency graph defined by
the matrix entries, achieving $O(N)$ setup cost.

\subsubsection{Convergence-gated hierarchy caching}

AMG hierarchy construction involves expensive sparse matrix--matrix products
(SpGEMM) for the Galerkin coarse-grid operators $A_c = R A P$. On static or
slowly varying meshes, the hierarchy changes minimally between consecutive
pressure solves. We introduce convergence-gated caching to amortize the
setup cost:

Let $k_i$ be the iteration count at solve~$i$, $\tau$ the maximum-iteration
threshold, and $N$ the cache interval. The hierarchy is rebuilt when
\begin{equation}
\label{eq:caching}
\mathrm{rebuild}(i) =
\begin{cases}
    \mathrm{true} & \text{if } i \bmod N = 0, \\
    \mathrm{true} & \text{if } k_{i-1} > \tau, \\
    \mathrm{false} & \text{otherwise.}
\end{cases}
\end{equation}
This amortizes the $O(N_{\mathrm{levels}} \cdot \mathrm{nnz})$ hierarchy
construction across $N$ solves while preserving convergence through the
iteration-count safety valve.

\subsection{Momentum system}
\label{sec:momentum}

The discretized momentum equation yields an asymmetric linear system
$A_U \mathbf{u} = \mathbf{b}_U$ due to the advection terms. We solve each
velocity component with the stabilized biconjugate gradient method
(BiCGStab).

\subsubsection{Block-Jacobi preconditioning}

Block-Jacobi extracts and inverts the block-diagonal of $A_U$, providing a
parallel preconditioner with minimal setup. Its quality degrades as mesh size
increases: the block-diagonal captures a shrinking fraction of the total
coupling.

\subsubsection{ILU-ISAI preconditioning}

For large meshes, we employ parallel incomplete LU (ParILU) factorization
following Chow and Patel~\cite{chow2015}: $A_U \approx LU$ computed via
iterative sweeps suitable for GPU execution. The triangular solves
$L^{-1}r$ and $U^{-1}r$ are then approximated using the Incomplete Sparse
Approximate Inverse (ISAI) technique~\cite{anzt2022}: precomputed sparse
matrices $\tilde{L}^{-1} \approx L^{-1}$ and $\tilde{U}^{-1} \approx
U^{-1}$ replace the sequential forward/backward substitution with parallel
sparse matrix--vector products:
\begin{equation}
M^{-1}r = \tilde{U}^{-1}\bigl(\tilde{L}^{-1}r\bigr).
\end{equation}
This transformation is essential for GPU performance: exact triangular solves
are inherently sequential (row~$i$ depends on row~$i{-}1$), achieving poor
thread occupancy. ISAI trades exactness for parallelism---the approximate
inverse is computed once and applied as two parallel \SpMV{} operations.

\subsection{LDU to CSR conversion}
\label{sec:ldu-csr}

OpenFOAM's face-based LDU storage must be converted to CSR for Ginkgo's GPU
kernels. We implement a two-phase approach:
\begin{enumerate}
    \item \textbf{Structure phase} (first solve only): Compute row pointers
        and column indices from the mesh adjacency. Cache for reuse on static
        meshes.
    \item \textbf{Value phase} (every solve): Copy diagonal and off-diagonal
        coefficients into the pre-allocated CSR value array. For symmetric
        matrices (pressure), upper and lower triangular values are mirrored.
        For asymmetric matrices (momentum), upper and lower values differ.
\end{enumerate}
The value refresh is triggered by \texttt{invalidateValues()}, ensuring the
CSR coefficients match the current PISO/PIMPLE corrector step without
redundant structure recomputation.
