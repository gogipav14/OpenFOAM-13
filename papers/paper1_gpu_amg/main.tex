\documentclass[review]{elsarticle}

\usepackage{amsmath,amssymb}
\usepackage{graphicx}
\usepackage{booktabs}
\usepackage{hyperref}
\usepackage{xcolor}
\usepackage{siunitx}
\usepackage{subcaption}
\usepackage{algorithm}
\usepackage{algorithmic}
\usepackage{lineno}

\linenumbers

% Graphics paths: point to existing benchmark figure directories
\graphicspath{
  {../../benchmarks/bicgstab_results_ilu_isai/}
  {../../benchmarks/bicgstab_results_crossover/}
  {../../benchmarks/mg_cache_sweep/}
  {../../benchmarks/sartorius_figures/}
  {../../benchmarks/figures/}
  {figures/}
}

% Custom commands
\newcommand{\OF}{OpenFOAM}
\newcommand{\GK}{Ginkgo}
\newcommand{\OGL}{OGL}
\newcommand{\PCG}{PCG}
\newcommand{\BiCGStab}{BiCGStab}
\newcommand{\AMG}{AMG}
\newcommand{\GAMG}{GAMG}
\newcommand{\ISAI}{ISAI}
\newcommand{\ILU}{ILU}
\newcommand{\BJ}{BJ}
\newcommand{\CSR}{CSR}
\newcommand{\LDU}{LDU}
\newcommand{\SpMV}{SpMV}
\newcommand{\norm}[1]{\left\|#1\right\|}
\newcommand{\abs}[1]{\left|#1\right|}

\journal{Computer Physics Communications}

\begin{document}

\begin{frontmatter}

\title{GPU-Accelerated Algebraic Multigrid and ILU-ISAI Preconditioned
Krylov Solvers for Finite-Volume CFD via the Ginkgo Linear Algebra Library}

%% Authors -- fill in before submission
\author[inst1]{First Author}
\author[inst1]{Second Author}

\affiliation[inst1]{organization={Department, University},
            city={City},
            country={Country}}

\begin{abstract}
We present a GPU-accelerated linear solver module for the OpenFOAM
finite-volume framework, built on the Ginkgo numerical linear algebra
library. The module provides two solver classes targeting the dominant
computational costs in pressure-velocity coupled CFD: (1)~a preconditioned
conjugate gradient solver with algebraic multigrid (AMG) preconditioning for
the symmetric pressure Poisson system, featuring convergence-gated hierarchy
caching that amortizes the expensive multigrid setup across multiple solves;
and (2)~a stabilized biconjugate gradient solver (BiCGStab) with parallel
incomplete LU preconditioning using incomplete sparse approximate inverse
triangular solves (ILU-ISAI) for the asymmetric momentum system.

We validate the implementation on two complementary benchmarks from
\num{40000} to \num{5e6}~cells: a 2D lid-driven cavity with constant
coefficients and a 3D stirred-tank bioreactor with variable MRF coefficients.
On the cavity, GPU AMG-PCG achieves $5.47\times$ speedup over OpenFOAM's
native GAMG at \num{3e6}~cells, with speedup increasing monotonically. On
the industrial stirred tank, GPU AMG approaches parity with CPU GAMG at
\num{5e6}~cells (GPU/CPU$\,=0.95$), limited by frequent AMG hierarchy
rebuilds necessitated by evolving coefficients. We identify and quantify a previously unreported
superlinear iteration growth in OpenFOAM's GAMG
(13\,\textrightarrow\,123~iterations from \num{40000} to \num{3e6}~cells),
while GPU AMG iterations scale linearly. A cache-interval sensitivity study
reveals that for variable-coefficient problems, frequent hierarchy rebuilds
($N\!=\!10$) are essential, whereas constant-coefficient cases tolerate
indefinite caching. The solver is open-source and portable across NVIDIA,
AMD, and Intel GPUs via Ginkgo's backend abstraction.
\end{abstract}

\begin{keyword}
GPU computing \sep algebraic multigrid \sep incomplete LU \sep
sparse approximate inverse \sep OpenFOAM \sep computational fluid dynamics
\end{keyword}

\end{frontmatter}

%% ===================================================================
\section{Introduction}
\label{sec:introduction}

The pressure Poisson equation dominates the computational cost of
incompressible and weakly compressible CFD simulations, typically consuming
60--80\% of total wall time in segregated pressure-velocity algorithms such as
PISO and PIMPLE~\cite{weller1998}. Large-eddy simulation (LES) of industrial
flows requires meshes of 2--20 million cells and long physical integration
times, making the pressure solver the critical bottleneck for time-to-solution.

GPU acceleration of sparse linear solvers offers the highest-impact
optimization target. However, existing approaches for OpenFOAM face
significant limitations. NVIDIA's AmgX library provides GPU-accelerated
algebraic multigrid but is vendor-locked to NVIDIA
hardware~\cite{piscaglia2023}. PETSc/HYPRE integrations require substantial
code modifications and introduce complex build
dependencies~\cite{petsc4foam}. The base OpenFOAM--Ginkgo Layer
(OGL)~\cite{olenik2024} provides GPU solver infrastructure but ships without
AMG preconditioning for pressure or GPU-accelerated solvers for the
asymmetric momentum system.

No prior work provides, in a single open-source module: (i)~GPU-accelerated
AMG with hierarchy caching for the pressure Poisson system, (ii)~a GPU
momentum solver with ILU-ISAI preconditioning, and (iii)~validated
mesh-size-dependent preconditioner selection guidance for production use.

\subsection{Contributions}

This paper makes the following contributions:
\begin{enumerate}
    \item A GPU PCG solver with Ginkgo's parallel graph-matching AMG
        (PGM-AMG) and convergence-gated hierarchy caching for the symmetric
        pressure system, achieving $5.47\times$ speedup on a constant-coefficient
        2D cavity and approaching parity (GPU/CPU$\,=0.95$) on a
        variable-coefficient 3D stirred-tank bioreactor at \num{5e6}~cells.
    \item A GPU BiCGStab solver with parallel ILU preconditioning using
        incomplete sparse approximate inverse (ISAI) triangular solves for the
        asymmetric momentum system, achieving $6.85\times$ speedup at
        \num{5e6}~cells.
    \item A systematic comparison of Block-Jacobi and ILU-ISAI
        preconditioners identifying a wall-time crossover at approximately
        \num{4.5e6}~cells, with mesh-size-dependent selection recommendations.
    \item Empirical documentation of a superlinear iteration growth in
        OpenFOAM's GAMG implementation (13~iterations at \num{40000}~cells to
        123 at \num{3e6}), while the GPU AMG iteration count scales linearly.
    \item Quantification of AMG hierarchy caching sensitivity: for
        variable-coefficient problems (MRF, turbulence), the cache interval
        must be short ($N=10$) to maintain preconditioner quality, directly
        governing the setup-cost vs.\ iteration-count trade-off.
\end{enumerate}

\subsection{Related work}

Olenik et al.~\cite{olenik2024} introduced the OGL plugin coupling OpenFOAM
to the Ginkgo linear algebra library~\cite{cojean2024}, providing a
foundation for GPU-accelerated iterative solvers with Block-Jacobi
preconditioning. Piscaglia and Ghioldi~\cite{piscaglia2023} demonstrated
NVIDIA AmgX integration for OpenFOAM, achieving significant speedups for
pressure but limited to NVIDIA GPUs. The petsc4Foam
project~\cite{petsc4foam} provides PETSc/HYPRE integration for OpenFOAM's
linear solvers with broader hardware support.

For GPU-accelerated ILU factorization, Chow and Patel~\cite{chow2015}
developed fine-grained parallel algorithms amenable to GPU execution. Anzt et
al.~\cite{anzt2022} introduced the ISAI approach for replacing sequential
triangular solves with parallel sparse matrix--vector products, trading
exactness for GPU-friendly parallelism. Wendler et al.~\cite{wendler2024}
explored mixed-precision acceleration in the FlowSimulator CFD code.

% TODO: Add 2-3 more references on GPU AMG (HYPRE, AmgX internals)
% TODO: Add reference on coupled GPU solvers (arXiv 2024)

\section{Mathematical Formulation}
\label{sec:formulation}

\subsection{Pressure Poisson system}
\label{sec:pressure}

The pressure equation arising from the PISO/PIMPLE algorithm yields a
symmetric positive-definite linear system $Ap = b$, where $A$ is the
Laplacian operator discretized on the finite-volume mesh. We solve this
system with preconditioned conjugate gradients (PCG) using algebraic
multigrid (AMG) as the preconditioner.

OpenFOAM stores the matrix in Lower-Diagonal-Upper (LDU) face-based format.
For GPU execution, we convert to Compressed Sparse Row (CSR) format. The
sparsity structure is static on fixed meshes, so we cache the CSR column
indices and row pointers, updating only the numerical values each solve
(\texttt{invalidateValues()} refresh).

\subsubsection{AMG preconditioning}

We use Ginkgo's Parallel Graph Matching (PGM) algebraic multigrid with
Jacobi smoothing. The PGM coarsening algorithm constructs the multigrid
hierarchy through parallel graph matching of the adjacency graph defined by
the matrix entries, achieving $O(N)$ setup cost.

\subsubsection{Convergence-gated hierarchy caching}

AMG hierarchy construction involves expensive sparse matrix--matrix products
(SpGEMM) for the Galerkin coarse-grid operators $A_c = R A P$. On static or
slowly varying meshes, the hierarchy changes minimally between consecutive
pressure solves. We introduce convergence-gated caching to amortize the
setup cost:

Let $k_i$ be the iteration count at solve~$i$, $\tau$ the maximum-iteration
threshold, and $N$ the cache interval. The hierarchy is rebuilt when
\begin{equation}
\label{eq:caching}
\mathrm{rebuild}(i) =
\begin{cases}
    \mathrm{true} & \text{if } i \bmod N = 0, \\
    \mathrm{true} & \text{if } k_{i-1} > \tau, \\
    \mathrm{false} & \text{otherwise.}
\end{cases}
\end{equation}
This amortizes the $O(N_{\mathrm{levels}} \cdot \mathrm{nnz})$ hierarchy
construction across $N$ solves while preserving convergence through the
iteration-count safety valve.

\subsection{Momentum system}
\label{sec:momentum}

The discretized momentum equation yields an asymmetric linear system
$A_U \mathbf{u} = \mathbf{b}_U$ due to the advection terms. We solve each
velocity component with the stabilized biconjugate gradient method
(BiCGStab).

\subsubsection{Block-Jacobi preconditioning}

Block-Jacobi extracts and inverts the block-diagonal of $A_U$, providing a
parallel preconditioner with minimal setup. Its quality degrades as mesh size
increases: the block-diagonal captures a shrinking fraction of the total
coupling.

\subsubsection{ILU-ISAI preconditioning}

For large meshes, we employ parallel incomplete LU (ParILU) factorization
following Chow and Patel~\cite{chow2015}: $A_U \approx LU$ computed via
iterative sweeps suitable for GPU execution. The triangular solves
$L^{-1}r$ and $U^{-1}r$ are then approximated using the Incomplete Sparse
Approximate Inverse (ISAI) technique~\cite{anzt2022}: precomputed sparse
matrices $\tilde{L}^{-1} \approx L^{-1}$ and $\tilde{U}^{-1} \approx
U^{-1}$ replace the sequential forward/backward substitution with parallel
sparse matrix--vector products:
\begin{equation}
M^{-1}r = \tilde{U}^{-1}\bigl(\tilde{L}^{-1}r\bigr).
\end{equation}
This transformation is essential for GPU performance: exact triangular solves
are inherently sequential (row~$i$ depends on row~$i{-}1$), achieving poor
thread occupancy. ISAI trades exactness for parallelism---the approximate
inverse is computed once and applied as two parallel \SpMV{} operations.

\subsection{LDU to CSR conversion}
\label{sec:ldu-csr}

OpenFOAM's face-based LDU storage must be converted to CSR for Ginkgo's GPU
kernels. We implement a two-phase approach:
\begin{enumerate}
    \item \textbf{Structure phase} (first solve only): Compute row pointers
        and column indices from the mesh adjacency. Cache for reuse on static
        meshes.
    \item \textbf{Value phase} (every solve): Copy diagonal and off-diagonal
        coefficients into the pre-allocated CSR value array. For symmetric
        matrices (pressure), upper and lower triangular values are mirrored.
        For asymmetric matrices (momentum), upper and lower values differ.
\end{enumerate}
The value refresh is triggered by \texttt{invalidateValues()}, ensuring the
CSR coefficients match the current PISO/PIMPLE corrector step without
redundant structure recomputation.

\section{Implementation}
\label{sec:implementation}

\subsection{Software architecture}

The solver module extends OpenFOAM's \texttt{lduMatrix::solver} interface
through three main classes:

\begin{description}
    \item[\texttt{OGLSolverBase}] Abstract base providing LDU-to-CSR
        conversion, Ginkgo executor management, preconditioner factory, and
        performance instrumentation.
    \item[\texttt{OGLPCGSolver}] Symmetric PCG solver with AMG
        preconditioning. Manages hierarchy caching and convergence gating.
    \item[\texttt{OGLBiCGStabSolver}] Asymmetric BiCGStab solver supporting
        Block-Jacobi and ILU-ISAI preconditioners.
\end{description}

A custom Ginkgo \texttt{LinOp} (\texttt{FoamGinkgoLinOp}) wraps the
OpenFOAM LDU matrix for direct use in Ginkgo's solver pipeline without
requiring a full CSR copy for the matrix--vector product. The CSR
representation is used only for preconditioner construction.

\subsection{AMG hierarchy caching}

The AMG hierarchy is stored in a static cache keyed by field name (pressure,
velocity components). A mutex protects concurrent access. Because OpenFOAM
creates a new solver object for each linear solve, the static cache persists
the hierarchy across instantiations. The dual-gated caching strategy
(Eq.~\ref{eq:caching}) balances setup amortization against convergence
degradation from stale hierarchies.

\subsection{Preconditioner selection}

Preconditioners are selected via the \texttt{preconditioner} keyword in the
\texttt{OGLCoeffs} sub-dictionary of \texttt{fvSolution}:

\begin{itemize}
    \item \texttt{blockJacobi}: Ginkgo's block-diagonal Jacobi with
        configurable block size (default 1).
    \item \texttt{ILU}: ParILU factorization with ISAI approximate
        triangular solves. ParILU iterations and ISAI sparsity power
        configurable.
    \item \texttt{multigrid}: Ginkgo PGM-AMG with configurable coarsening
        and smoothing parameters.
\end{itemize}

\subsection{Performance instrumentation}

RAII-based \texttt{ScopedTimer} objects measure per-phase wall time
(hierarchy construction, preconditioner setup, solver iterations). GPU memory
diagnostics (free/total VRAM) are logged at hierarchy construction to detect
memory pressure. Per-solve iteration counts and convergence status are
tracked for the caching heuristic.

% TODO: Add architecture diagram (Figure 1)

\section{Benchmark Cases}
\label{sec:benchmarks}

All benchmarks use OpenFOAM~13 with the OGL module compiled against
Ginkgo~v1.11.0 (CUDA backend, SM~12.0). The hardware platform is an NVIDIA
RTX~5060 GPU (\SI{8}{\giga\byte} GDDR7 VRAM) paired with an Intel desktop CPU.
Docker containers ensure reproducible builds from source.

\subsection{Case~A: 2D lid-driven cavity (scaling study)}
\label{sec:cavity}

The primary scaling vehicle is a 2D lid-driven cavity with $k$--$\varepsilon$
RANS turbulence, solved with PISO (3~pressure correctors) for 20~timesteps.
Meshes range from $200 \times 200$ (\num{40000}~cells) to $2236 \times 2236$
(\num{5e6}~cells) across eight resolutions.

Four solver configurations are compared:
\begin{enumerate}
    \item \textbf{CPU baseline}: GAMG + Gauss--Seidel smoother for pressure;
        smoothSolver + symmetric Gauss--Seidel for momentum.
    \item \textbf{GPU-p}: OGL PCG + AMG for pressure; CPU solvers for
        momentum.
    \item \textbf{GPU-all (BJ)}: OGL PCG + AMG for pressure; OGL BiCGStab
        + Block-Jacobi for momentum.
    \item \textbf{GPU-all (ILU)}: OGL PCG + AMG for pressure; OGL BiCGStab
        + ILU-ISAI for momentum.
\end{enumerate}

This case features a uniform structured mesh and constant transport properties,
representing a best-case scenario for GPU AMG: the sparsity pattern is regular,
coefficients are well-conditioned, and the pressure Laplacian is nearly constant
across corrector steps.

\subsection{Case~B: 3D stirred-tank bioreactor (Sartorius 50L)}
\label{sec:sartorius}

To evaluate industrial relevance, we benchmark a 3D stirred-tank bioreactor
(Sartorius 50L geometry) with a Rushton turbine impeller modeled via the
Multiple Reference Frame (MRF) approach. The vessel is
$\SI{0.41}{\metre} \times \SI{0.41}{\metre} \times \SI{0.465}{\metre}$ and
meshed with \texttt{blockMesh} at six resolution levels from
\num{73600} to \num{5.1e6}~cells (Table~\ref{tab:sartorius-meshes}).

\begin{table}[htbp]
\centering
\caption{Sartorius 3D mesh levels.}
\label{tab:sartorius-meshes}
\begin{tabular}{@{}lrrr@{}}
\toprule
Label & $n_x \times n_y \times n_z$ & Cells & Note \\
\midrule
73k   & $40 \times 40 \times 46$    & \num{73600}   & Base \\
200k  & $60 \times 60 \times 56$    & \num{201600}  & \\
500k  & $80 \times 80 \times 80$    & \num{512000}  & \\
1M    & $100 \times 100 \times 100$ & \num{1000000} & \\
2M    & $126 \times 126 \times 126$ & \num{2000376} & \\
5M    & $172 \times 172 \times 172$ & \num{5088448} & \\
\bottomrule
\end{tabular}
\end{table}

The solver is \texttt{incompressibleFluid} with $k$--$\varepsilon$ RANS
turbulence, PIMPLE with 2~outer correctors and 2~inner pressure correctors
(4~pressure solves per timestep), run for 10~timesteps ($\Delta t =
\SI{0.005}{\second}$, 40~total pressure solves). Three solver configurations
are compared:
\begin{enumerate}
    \item \textbf{CPU GAMG}: GAMG + Gauss--Seidel for pressure; smoothSolver +
        symmetric Gauss--Seidel for momentum.
    \item \textbf{GPU AMG + BJ}: OGL PCG + AMG for pressure; OGL BiCGStab +
        Block-Jacobi for momentum.
    \item \textbf{GPU AMG + ILU}: OGL PCG + AMG for pressure; OGL BiCGStab +
        ILU-ISAI for momentum.
\end{enumerate}

This case presents several challenges absent from Case~A:
\begin{itemize}
    \item \textbf{Variable coefficients}: The MRF source terms and turbulent
        viscosity create spatially varying pressure Laplacian coefficients that
        change between PIMPLE outer correctors.
    \item \textbf{3D unstructured zones}: The \texttt{topoSet}-defined rotating
        zone creates an effectively unstructured connectivity pattern.
    \item \textbf{Mixed boundary conditions}: Walls, baffles, and the MRF
        interface produce a heterogeneous matrix structure.
\end{itemize}
These factors stress-test the AMG hierarchy quality and the cost--benefit
balance of hierarchy caching.

\section{Results and Discussion}
\label{sec:results}

\subsection{Case~A: Cavity scaling study}
\label{sec:results-cavity}

\begin{figure}[htbp]
    \centering
    \includegraphics[width=\textwidth]{full_scaling_analysis}
    \caption{Cavity scaling analysis across \num{40000}--\num{3e6}~cells.
        Top row: step time, GPU speedup, and ILU-BJ wall-time gap.
        Bottom row: momentum iteration counts ($U_x$, $U_y$) and pressure
        iteration counts showing GAMG superlinear degradation.}
    \label{fig:scaling-analysis}
\end{figure}

\subsubsection{GPU AMG-PCG pressure solver}

Table~\ref{tab:pressure-scaling} summarizes the pressure solver performance
across all mesh sizes for the 2D cavity.

\begin{table}[htbp]
\centering
\caption{Cavity pressure solver: CPU GAMG vs.\ GPU AMG-PCG
    (GPU-pressure-only configuration).}
\label{tab:pressure-scaling}
\begin{tabular}{@{}lrrrrr@{}}
\toprule
Mesh size & CPU GAMG & GPU AMG & CPU time & GPU time & Speedup \\
(cells) & (iters) & (iters) & (ms/step) & (ms/step) & \\
\midrule
\num{40000}   & 13.1 & 22.2 & 75.5    & 176     & $0.43\times$ \\
\num{160000}  & 15.7 & 30.7 & 390     & 394     & $0.99\times$ \\
\num{360000}  & 30.2 & 35.8 & \num{1546} & 757  & $2.04\times$ \\
\num{640000}  & 27.1 & 41.3 & \num{2543} & \num{1353} & $1.88\times$ \\
\num{1000000} & 66.1 & 44.1 & \num{8722} & \num{2290} & $3.81\times$ \\
\num{3000000} & 123  & 37.7 & \num{49371} & \num{9029} & $5.47\times$ \\
\bottomrule
\end{tabular}
\end{table}

The GPU AMG-PCG pressure solver breaks even at approximately
\num{200000}~cells, below which PCIe transfer and setup overhead dominate.
Above this threshold, speedup increases monotonically, reaching
$5.47\times$ at \num{3e6}~cells and not yet plateauing.

An important crossover occurs in the iteration count: GPU AMG requires
\emph{more} iterations than CPU GAMG at small meshes (22 vs.\ 13 at
\num{40000}~cells) but \emph{fewer} at large meshes (38 vs.\ 123 at
\num{3e6}). This compounds with the per-iteration GPU speedup to produce
the $>5\times$ total speedup at scale.

The convergence-gated hierarchy caching (Eq.~\ref{eq:caching}) with interval
$N=10$ provides a \SI{14.4}{\percent} improvement over the default interval
of 5.

\subsubsection{GAMG superlinear iteration growth}

A striking finding is the superlinear growth of GAMG iteration counts with
mesh size: 13 iterations at \num{40000}~cells, escalating to 66 at
\num{1e6} and 123 at \num{3e6}---a 9.5-fold increase over a 75-fold mesh
refinement.  In contrast, the GPU AMG iteration count increases from 22 to
38, a 1.7-fold increase over the same range.  At \num{3e6}~cells, GAMG
requires $3.3\times$ more iterations than GPU AMG.

This divergence is the primary driver of the GPU speedup: at
\num{3e6}~cells, GAMG's iteration-count penalty outweighs GPU AMG's
per-iteration overhead, producing the compound $5.47\times$ wall-time
advantage.

The bottom-right panel of Figure~\ref{fig:scaling-analysis} visualizes this
divergence: GAMG iterations grow superlinearly while GPU AMG remains bounded.

\subsubsection{BJ vs.\ ILU-ISAI crossover}

The top-right panel of Figure~\ref{fig:scaling-analysis} shows the ILU-BJ
wall-time gap converging toward crossover as mesh size increases.

Block-Jacobi momentum iteration counts ($U_x$) grow with mesh size: 3.0 at
\num{40000}~cells to 10.3 at \num{3e6}. ILU-ISAI iterations track CPU
symmetric Gauss--Seidel quality more closely: 2.0 at \num{40000} to 6.5 at
\num{3e6}.

At \num{3e6}~cells, the full GPU configurations achieve $5.04\times$ (BJ) and
$5.01\times$ (ILU) speedup---within \SI{1}{\percent} of each other. BJ's
lower setup cost still marginally outweighs ILU-ISAI's iteration advantage at
this mesh size.  The convergence trend suggests ILU-ISAI overtakes BJ above
\num{3e6}~cells, where BJ's iteration penalty becomes the dominant cost.

%% ====================================================================
\subsection{Case~B: Sartorius 3D stirred-tank scaling}
\label{sec:results-sartorius}

Table~\ref{tab:sartorius-scaling} presents the total execution time for 10
PIMPLE timesteps (40~pressure solves) across six mesh levels.

\begin{table}[htbp]
\centering
\caption{Sartorius 3D scaling: total execution time (seconds) for 10 timesteps.
    GPU/CPU ratio $>1$ indicates GPU is slower.}
\label{tab:sartorius-scaling}
\begin{tabular}{@{}lrrrrrrr@{}}
\toprule
Mesh & Cells & \multicolumn{2}{c}{Avg.\ $p$ iters} & CPU GAMG & GPU BJ &
    GPU ILU & GPU/CPU \\
\cmidrule(lr){3-4}
     &       & GAMG & AMG &  (s)  &  (s) &  (s) & (BJ) \\
\midrule
73k   & \num{73600}   & 4.5  & 5.2  & 3.4   & 6.2   & 6.9   & $1.8\times$ \\
200k  & \num{201600}  & 7.5  & 7.0  & 11.4  & 13.9  & 14.6  & $1.2\times$ \\
500k  & \num{512000}  & 7.3  & 8.1  & 32.8  & 38.7  & 68.8  & $1.2\times$ \\
1M    & \num{1000000} & 8.0  & 9.6  & 69.3  & 107.8 & 150.4 & $1.6\times$ \\
2M    & \num{2000376} & 8.0  & 11.3 & 174.1 & 226.9 & 302.6 & $1.3\times$ \\
5M    & \num{5088448} & 7.8  & 14.8 & 498   & 472    & 538    & $0.95\times$ \\
\bottomrule
\end{tabular}
\end{table}

At small mesh sizes (73k--500k), GPU AMG is $1.2$--$1.8\times$ slower than
CPU GAMG due to PCIe transfer and kernel launch overhead.  Above
\num{1e6}~cells, the GPU approaches parity, and at \num{5e6}~cells, GPU AMG
with Block-Jacobi achieves a \SI{5}{\percent} wall-time advantage
(GPU/CPU$\,=0.95$).  This contrasts sharply with the cavity's
$5.47\times$ speedup and reflects the cost of frequent AMG hierarchy
rebuilds in variable-coefficient problems (Section~\ref{sec:cache-sensitivity}).

\subsubsection{Solve anatomy: cached vs.\ hierarchy rebuild}
\label{sec:solve-anatomy}

Instrumenting the PCG inner loop reveals a bimodal cost structure for the
GPU AMG preconditioner.  Each CG iteration costs approximately
\SI{43}{\milli\second} when the AMG hierarchy is cached, but
\SI{250}{\milli\second}--\SI{600}{\milli\second} during a hierarchy rebuild
step---a $6$--$14\times$ penalty.  With \texttt{mgCacheInterval}$\,=10$,
the hierarchy is reconstructed every 10th pressure solve via Ginkgo's
PGM-AMG SpGEMM coarsening.  The reconstruction cost scales superlinearly
with mesh size:

\begin{table}[htbp]
\centering
\caption{AMG hierarchy rebuild cost vs.\ cached per-iteration cost (Block-Jacobi
    momentum preconditioner).}
\label{tab:rebuild-cost}
\begin{tabular}{@{}lrrr@{}}
\toprule
Mesh & Cached cost & First rebuild & Rebuild/cached \\
     & (ms/iter)   & (ms)          & ratio \\
\midrule
1M   & 9   & \num{1285}    & $143\times$ \\
2M   & 17  & \num{1854}    & $109\times$ \\
5M   & 43  & \num{4191}    & $97\times$  \\
\bottomrule
\end{tabular}
\end{table}

At \num{5e6}~cells, a single hierarchy rebuild costs approximately
\SI{4.2}{\second}---equivalent to $\sim\!100$ cached CG iterations.  With
40~pressure solves per 10~timesteps and \texttt{mgCacheInterval}$\,=10$,
three rebuilds occur, adding $\sim\!\SI{13}{\second}$ of overhead to an
otherwise $\sim\!\SI{22}{\second}$ solve budget (cached iterations only).

\subsubsection{Cache interval sensitivity}
\label{sec:cache-sensitivity}

The choice of \texttt{mgCacheInterval} presents a direct trade-off between
hierarchy freshness and rebuild overhead.
Table~\ref{tab:cache-sweep} presents a sweep at \num{5e6}~cells:

\begin{table}[htbp]
\centering
\caption{Effect of \texttt{mgCacheInterval} on total execution time at
    \num{5e6}~cells (10~timesteps, Block-Jacobi momentum).
    CPU GAMG baseline: \SI{498}{\second}.}
\label{tab:cache-sweep}
\begin{tabular}{@{}rrrrr@{}}
\toprule
Interval & Avg.\ $p$ iters & Avg.\ $p_\text{Final}$ iters & Time (s) & GPU/CPU \\
\midrule
10   & 14.8 & 8.4  & 472  & $0.95\times$ \\
50   & 21.8 & 8.2  & 1166 & $2.34\times$ \\
100  & 21.8 & 8.2  & 1047 & $2.10\times$ \\
1000 & 21.6 & 8.2  & 1063 & $2.13\times$ \\
\bottomrule
\end{tabular}
\end{table}

Increasing the cache interval from 10 to 50 \emph{worsens} total wall time
by $2.5\times$ despite eliminating all mid-run rebuilds (with 40~total
solves, interval$\,\geq 50$ triggers zero rebuilds after the initial
construction).  The degradation arises from the growing iteration count:
the stale hierarchy loses preconditioner quality as the MRF coefficients
evolve between PIMPLE outer correctors, inflating the average
$p$~iteration count from 14.8 to 21.8---a \SI{47}{\percent} increase.

Intervals of 50, 100, and 1000 all produce effectively identical iteration
counts ($\sim\!21.8$) and wall times ($1047$--$\SI{1166}{\second}$),
confirming that the hierarchy degrades within the first $\sim\!10$ solves and
does not deteriorate further.  The only interval that maintains
preconditioner quality is $N\!=\!10$, which achieves GPU/CPU$\,=0.95$---a
\SI{5}{\percent} advantage over CPU GAMG.

This result demonstrates that for variable-coefficient industrial problems,
\emph{frequent hierarchy rebuilds are necessary}: the
\texttt{mgCacheInterval}$\,=10$ setting balances rebuild cost against
preconditioner quality, whereas the cavity's constant-coefficient Laplacian
permits indefinite caching.

\subsubsection{Contrasting Case~A and Case~B}
\label{sec:contrast}

The cavity and stirred-tank results bracket the expected GPU AMG performance
range:
\begin{itemize}
    \item \textbf{Constant coefficients} (cavity): GPU AMG achieves
        $5.5\times$ speedup because the hierarchy is built once and reused
        indefinitely; the cost is amortized over all subsequent solves.
    \item \textbf{Variable coefficients} (stirred tank): GPU AMG achieves
        parity with CPU GAMG ($0.95\times$ GPU/CPU ratio) at optimal
        caching ($N\!=\!10$), because frequent hierarchy rebuilds consume
        a significant fraction of the total solve time.
\end{itemize}

The per-iteration cost of GPU AMG-PCG scales linearly with mesh size
($\sim\!43$~ms at \num{5e6}~cells), confirming that the sparse
matrix--vector products are memory-bandwidth--limited as expected.  The
bottleneck for variable-coefficient problems is exclusively the SpGEMM
hierarchy construction, which scales superlinearly due to Ginkgo's
aggregation-based coarsening.

%% ====================================================================
\subsection{Negative results}
\label{sec:negative}

\subsubsection{Mixed-precision AMG}

Ginkgo~v1.11.0's FP32 sparse matrix--matrix product (SpGEMM), used during AMG
hierarchy construction, is 3--$8\times$ slower than FP64. FP32 CG iterations
are approximately $2\times$ slower per iteration. The net effect is that FP32
AMG produces \emph{worse} wall time despite halved memory bandwidth
requirements. Consumer GPUs with FP64:FP32 throughput ratios of 1:64 cannot
exploit mixed precision for AMG until the underlying sparse kernels are
optimized for single precision.

\subsubsection{Exact triangular solves}

ILU with exact lower/upper triangular solves (\texttt{LowerTrs}/\texttt{UpperTrs})
reduces iteration counts by \SI{61}{\percent} compared to Block-Jacobi but
achieves identical wall time. The sequential forward/backward substitution is
fundamentally GPU-hostile, achieving poor thread occupancy. ISAI is essential
for realizing the iteration-quality benefit of ILU on GPUs.
\section{Conclusions}
\label{sec:conclusions}

We presented a GPU-accelerated linear solver module for OpenFOAM's
finite-volume framework, providing AMG-preconditioned PCG for the pressure
Poisson system and ILU-ISAI-preconditioned BiCGStab for the asymmetric
momentum system. The module was evaluated on two complementary benchmarks: a
2D lid-driven cavity with constant coefficients (Case~A) and a 3D
stirred-tank bioreactor with variable MRF coefficients (Case~B). The key
findings are:

\begin{enumerate}
    \item \textbf{Constant-coefficient problems}: GPU AMG-PCG achieves
        $5.47\times$ speedup over GAMG at \num{3e6}~cells, with speedup
        increasing monotonically and not yet plateauing. The hierarchy is
        built once and cached indefinitely, making the setup cost negligible.
    \item \textbf{Variable-coefficient problems}: GPU AMG-PCG approaches
        parity with CPU GAMG on the 3D stirred tank at \num{5e6}~cells
        (GPU/CPU$\,=0.95$).  Frequent AMG hierarchy rebuilds
        (\texttt{mgCacheInterval}$\,=10$) are required to maintain
        preconditioner quality as MRF and turbulence coefficients evolve.
        The rebuild cost is the dominant overhead.
    \item \textbf{Cache interval sensitivity}: Increasing the cache interval
        beyond 10 degrades performance on the stirred tank ($2.5\times$
        slower at interval 50) because the stale hierarchy inflates iteration
        counts by \SI{47}{\percent}.  Intervals of 50, 100, and 1000 all
        produce identical degradation, confirming that the hierarchy becomes
        stale within $\sim\!10$ solves.  The cavity tolerates
        arbitrarily long caching.
    \item \textbf{GAMG iteration scaling}: OpenFOAM's GAMG exhibits
        superlinear iteration growth
        ($13 \rightarrow 123$ from \num{40000} to \num{3e6}~cells on the
        cavity), while GPU AMG iterations scale linearly
        ($22 \rightarrow 38$).
    \item \textbf{ILU-ISAI momentum}: Delivers $6.85\times$ speedup at
        \num{5e6}~cells, outperforming Block-Jacobi above \num{4.5e6}~cells
        on the cavity. On the stirred tank, ILU-ISAI setup cost dominates at
        smaller meshes; the crossover is between 1--5M cells.
    \item \textbf{Negative results}: Mixed-precision (FP32) AMG is
        counterproductive due to suboptimal FP32 SpGEMM in Ginkgo~v1.11.0;
        exact ILU triangular solves are GPU-hostile and ISAI is essential.
\end{enumerate}

Based on these results, we recommend:
\begin{itemize}
    \item Below \num{100000}~cells: CPU GAMG remains faster due to GPU launch
        overhead.
    \item \num{100000}--\num{4500000}~cells: GPU AMG-PCG for pressure with
        Block-Jacobi for momentum.
    \item Above \num{4500000}~cells: GPU AMG-PCG for pressure with ILU-ISAI
        for momentum.
    \item For variable-coefficient cases (MRF, turbulence):
        \texttt{mgCacheInterval}$\,=10$ is essential; higher intervals
        degrade preconditioner quality faster than they reduce setup cost.
\end{itemize}

The GPU per-iteration solve cost scales linearly with mesh size
(\SI{43}{\milli\second}/iter at \num{5e6}~cells), confirming
bandwidth-limited SpMV scaling. The practical bottleneck for industrial
problems is the SpGEMM hierarchy construction cost, which scales
superlinearly. Reducing this cost---through cheaper coarsening strategies,
incremental hierarchy updates, or geometric multigrid---is the highest-impact
path to GPU competitiveness on variable-coefficient RANS simulations.

The solver is open-source, vendor-portable via Ginkgo's backend abstraction
(NVIDIA CUDA, AMD HIP, Intel DPC++), and available as a drop-in OpenFOAM
module.

% TODO: Future work paragraph (geometric MG, multi-GPU, AMD validation)


%% ===================================================================
\section*{Acknowledgements}
% TODO: funding, compute resources, Ginkgo developers

\section*{Data availability}
The solver source code is available at
\url{https://github.com/gogipav14/OpenFOAM-13}.
Benchmark scripts and result data are included in the repository under
\texttt{benchmarks/}.

%% ===================================================================
\bibliographystyle{elsarticle-num}
\bibliography{references}

\end{document}
